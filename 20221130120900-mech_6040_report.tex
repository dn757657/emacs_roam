% Created 2022-12-04 Sun 14:43
% Intended LaTeX compiler: pdflatex
\documentclass[11pt]{article}
\usepackage[utf8]{inputenc}
\usepackage[T1]{fontenc}
\usepackage{graphicx}
\usepackage{longtable}
\usepackage{wrapfig}
\usepackage{rotating}
\usepackage[normalem]{ulem}
\usepackage{amsmath}
\usepackage{amssymb}
\usepackage{capt-of}
\usepackage{hyperref}
\date{\today}
\title{mech 6040 report}
\hypersetup{
 pdfauthor={},
 pdftitle={mech 6040 report},
 pdfkeywords={},
 pdfsubject={},
 pdfcreator={Emacs 28.1 (Org mode 9.5.2)}, 
 pdflang={English}}
\begin{document}

\maketitle
\tableofcontents


\section{Formatting}
\label{sec:org0fdc752}
\begin{itemize}
\item 10-15 page limit
\item title page
\item 8-12 body pages
\item 1 ref page
\end{itemize}

\section{Title Page}
\label{sec:org9a132c6}
\begin{itemize}
\item Course
\item Project Title
\item Figure
\item Student Name
\item Date \& Signature?
\end{itemize}

\section{Main Body}
\label{sec:org818f31c}
\begin{itemize}
\item walk reader through
\item dos and donts - present within each component blurb
\item problems and solutions - present wihtin each component blurb
\end{itemize}

\subsection{Formatting}
\label{sec:org3a257dc}
\begin{itemize}
\item 10-13 pages
\item 1000-2000 word
\item 10-12 FIgures: 1
\end{itemize}

\subsection{Figures}
\label{sec:org4f83416}
\begin{enumerate}
\item exploded diagram
\item flap close up with axle relief labelled maybe other things also
\item flap wheel shwoing min flow thickness
\item title page photo
\item rotating assembly
\item shell assembly
\item drivetrain
\item LEGO axle profile - zotero
\item comparative typical/production split flap clock
\item flaps both sides one figure (showcase dual material prints)
\item worm alone to showcase finish quality and tooth bias? coudl be two
\end{enumerate}

\subsection{{\bfseries\sffamily TODO} }
\label{sec:org3704c6a}
\begin{itemize}
\item reference teds lego axle profile and include drawing
\item units from Doman template
\item worm video reference
\item bevel gear video reference
\item notes on how parts were printed? if its important to the design
\item reference all components to named item in exploded drawing
\item add names of components used in report to exploded drawing
\end{itemize}

\subsection{Assembly}
\label{sec:orgf43d9c0}
\begin{itemize}
\item great place for exploded diagram idenifying components
\item lego axel lengths!
\end{itemize}

\subsubsection{Rotating Assembly}
\label{sec:org744138d}
\begin{enumerate}
\item slide one flap wheel onto XXmm LEGO axle
\item assemble second flap wheel and ensure both flap wheel holes align by temprarily mating both components, note this orientation
\item remove second flap wheel
\item insert central spacer between both flap wheels and replace second flap wheel
\item insert individual flaps by exploiting flap compliance, snap flaps into place
\item assemble the 11mm spacer outside the flap wheel assembly, followed by the worm gear
\item assemble a 1mm spacer on the opposing side of the flap wheel assembly
\item to complete the rotating assembly, assemble a 2mm spacer outside the worm gear
\end{enumerate}

\subsubsection{Shell}
\label{sec:orgbea2bad}
\begin{enumerate}
\item assemble the round bevel gear mount and worm gear mounts by pressing into the right shell half as shown in [ref exploded]
\item assemble the worm, straddled by two 1mm spacers between both worm mounts and peirce with a XXmm LEGO axle
\item stack the remaining bevel gear, 1mm spacer, bevel mount and handle, as shown in [ref exploded]
\item peirce the stacked components with a XXmm LEGO axle
\item press the bevel gear assembly into the shell cut-outs above and below the round bevel gear mount
\item assemble the 4mm spacer and one bevel gear respectively to the worm axle end nearest the bevel gear mount, ensuring bevel gears mesh
\end{enumerate}

\subsubsection{Final}
\label{sec:org1f4616a}
\begin{enumerate}
\item insert axle end or rotating assembly with worm follower gear into the centre hole in right side shell, meshing worm and worm gear
\item assmeble left shell half to opposing end of rotating assembly shaft
\item press connectors into top and bottom, mating shell halves
\item insert top (long) and bottom (short) stops into shell cutouts near top and bottom
\end{enumerate}

\subsection{Component Design}
\label{sec:org40e3b31}
The driving design factors for this split flap display assembly are:
\begin{itemize}
\item minimize overall size, maximize flap display area
\item ensure displays can be assembled in a modular array for use as a time keeping device
\item smooth operation
\item consistent flap actuation
\end{itemize}

All component faces interfacing with the build plate incoporate a small chamfer to improve finish quality and minimize effects of first layer deformation. Refer to [ref exploded diagram] for component visuals.

\subsubsection{Fits}
\label{sec:org13a214d}
Wihtin the following component design breakdowns are various references concerning part fits. These references are defined in the following tables [ref] and [ref] for use throughout the report. The offset column in table [ref] indicates the difference in dimensions for interfacing features, such that the hole feature is always larger or equal in size to the mating feature. Similarly in table [ref] the offset column indicates the amount added to the stock LEGO axle profile [ref], in creating LEGO axle hole features. 

Fits for 3D Printed Components
\begin{center}
\begin{tabular}{lrl}
Fit Type & Offset [mm] & Description\\
\hline
Semi-permanent & 0 0 & Components can be assembled by hand with significant force applied, dissasmbly may require tools and result in permanent deformation\\
Tight & 0.05-0.1 & Component can be assembled and dissassembled by hand with moderate force applied\\
Sliding & 0.02-0.03 & Components can move freely relative to one another  freely\\
\end{tabular}
\end{center}

Fits for LEGO Interfacting Components
\begin{center}
\begin{tabular}{lrl}
Fit Type & Offset [mm] & Description\\
\hline
Fixed & 0.05-0.1 & Component does not slide easily on LEGO axle\\
Intermediate & 0.1-0.2 & Component does not slide freely, but will slide easily on LEGO axle\\
Sliding & 0.2+ & Component slides freely on LEGO axle\\
\end{tabular}
\end{center}


\subsubsection{Roating Split Flap Assembly}
\label{sec:org8609d8a}
The key driving factor of the assembly design lies within the rotating split flap assembly. Aspects of both the flaps, and the wheels housing the flaps limit the size and shape of the entire assembly. Critical design elements, challenges and solutions are outlined below.

\begin{enumerate}
\item Flaps
\label{sec:orgd389e61}

The flaps are designed to be printed in multiple layers, comprising of black top and bottom layers with display numbers relieved,  three white core layers offering maximum readiblity and contrast. Each flap face where the relieved number area is minimized, is oriented face down on the build plate for ideal printing. The final flap designs vertical to horizontal aspect ratio is driven by the chosen font.

The most challenging flap component design constraint is limiting thickness. The dominant driver of flap thickness is the relationship with the distance between flap mounting holes. As the flaps are displayed, they must achieve a vertical position, given that the preceeding flap is also near vertical, the horizontal distance between flap mounting holes must accomodate two flaps oriented vertically and stacked face to face [ref]. Maxmizing flap thickness within this limitation improves flap axle durability, and the appearance and contrast of each flap digit. By reducing the layer height to 0.1mm, aside from a 0.2mm first layer for bed adhesion, the final flap design comprises 2 peripheral black, and 3 white core layers. Achieving satisfactory flap appearance given the limited thickness window. Growing flap thickness also implies a trade-off with flap wheel diameter, as the flap thickness is also the flap axle width, and flap axle height cannot be reduced further than 2 extrusion widths, imposed by the PrusaSlicer.

The ideal split flap display minizes the central split clearance between vertically displayed flaps. This distance was iteratively optimized by observing flap interference given various central split clearances.To ensure smooth actuation, each sucessive flap must be nearly vertical when the upper display flap falls, leaving it exposed. Any additional negative angle relative to the rotating direction will result in a suboptimal viewing angle for each successive flap. The height of the flap relief where the flap wheel is mounted must be minimized to maximize the verticality of each new flap displayed. An optimal geometry has been achived by incorporating zero clearance between the flap relief [ref] and flap moutning wheel when the flaps are oriented vertically.

\item Flap Wheels
\label{sec:org38a7c3d}

\begin{enumerate}
\item Final Design
\label{sec:org011b345}
There are thirty flaps in the final design as assemblies with lower flap counts exhibited dissatisfying flap actuation delay. Thirty is a common multiple of both ten and six, allowing the display to be reused in a time keeping array. LEGO axle hole fit offset is tight to facilitate ease of flap and flap wheel assembly..

\item Design Intent
\label{sec:orgcfd8c69}
The ideal flap wheel design minimizes flap wheel size, in turn maximizing the flap display area, and allows the flaps to rotate freely wihtin thier respective mounting holes. The flap wheel is critical to achieving smooth and consistent actuation of the split flap display. Designs progressed from thick flaps, and few flap mounting holes, to minimal optimized flap thickness and many flap mounting holes. Increasing the number of flap mounting holes, and subsequently the number of flaps, ensures the delay between the top flap rotating to the bottom flap position, and the next top flap in line reaching a vertical position decreases, resulting in smoother operation.

\item Design Limitations
\label{sec:org48aa029}
The proximity of flap moutning holes to one another is limited by nozzle size. As a minimum of one perimeter must be included  for each component feature. The flap mounting hole size is limited by the flap axle size and flap thickness, as is described above in section [ref - Flaps design limitations].
\end{enumerate}

\item Central Spacer
\label{sec:org95e4e41}
The central spacer performs a critical role in creating smooth flap actuation. The centre spacer sets the distance between the opposing flap mounting wheels, ultimately creating clearance between the flap edges and flap wheels essential to free flap actuation. Five different length spacers were trialed, begginning with the nominal axle distance between flaps, and increasing in 0.1mm increments. The smallest of the spacers was selected for the final design, as flaps could actuate freely without binding, therefore additional flap clearance was not required. The central spacer LEGO axle hole featiure is designed with a sliding fit.

\item Worm Follower Gear
\label{sec:orga0bf3bc}
The final worm gear design is a thin spur gear with a modulus of one, and thirty teeth. Thirty teeth creates a 30:1 gear ratio relative to the roational input and split flap assembly, resulting in one flap acutation per handle rotation. This simplified ratio increases unit modularity. To prevent the need for a specialized worm follower gear, the thickness of the gear is limited, gears beyond a thickness of 2mm exhibited elevated binding, while a 2mm thick gear minimized drivetrain backlash. The gear teeth clearance, defined as the amount symmetrically relived from each tooth, and the mounting clearance, defined as the amount subtracted from the gear pitch diameter are itteratively reduced to achieve an optimal balance between system backlash and required input force. The final worm follower gear is designed with a tight LEGO fit axle hole feature for easy of assembly.
\end{enumerate}

\subsubsection{Drivetrain}
\label{sec:org6164106}
Each drivetrain component is selected and designed to maximize smooth, reliable device actuation. A worm and worm gear drivetrain is selected as a large gear ratio is ideal for a time keeping device. The worm gear drive is coupled with two bevel gears to facilitate easy hand actuation of the split flap display.

\begin{enumerate}
\item Worm
\label{sec:orgeb250ab}
A worm with a modulus of one was designed using an online tutorial [ref]. The chosen length of the final design offers sufficient engagement with the worm gear, adding minimal backlash to the overall drivetrain. A sliding fit offset is applied to the central LEGO axle hole feature. No clearances are introduced in the final worm design, as challenging prints limited itteration. 

To eliminate the need for support material and improve the surface quality of the worm teeth and axial hole feature, the worm rotational axis is oriented vertically for printing. This orientation introduces build plate adherance issues, therefore a small brim is added to the base. The worm gear teeth overhang significantly in this orientaiton. Although the teeth overhang did not intropduce print failure, it is observed that as teeht are built vertically, each tooth tip curls upwards, resulting in an upwards bias of each completed tooth. This did not significantly affect worm performance. The layer height is reduced to facilitate smooth operation of the worm and worm gear drive, and to assist with overhang performance.

\item Thrust bearings/spacers
\label{sec:org76254bb}
Each drivetrain compoent is seperated by small spacers. Spacers serve multiple functional purposes such as reducing roating friction by mimizing the contact areas of rotational interfaces, and for precisely aligning drivetrain components. Spacers are designed with a sliding fit offset for easy assembly. Spacers exceeding 10mm incorporate a small brim. Each 3D printed component exhibits both a smooth and rough side, this phenomena allows spacers to also serve as simple thrust bearings by creating additional smooth rotational interfaces within rotating assemblies.

\item Bevels
\label{sec:org6ae96ba}
The base bevel gear shape is modelled using an instructional video [ref]. The base tooth geometry is modified by adding a small chamfer to both circumferential edges of each tooth. The chamfers limit overhangs during print, improve final part quality and facilitate smooth bevel gear engagement. A 24 tooth bevel gear design is chosen for the final componnent, this is the minimum size whithout eliminating the inner bevel gear support, which is essential as is described below in section [ref section about bevel mounting]. The LEGO axle interfacing feature employs an intermediate fit offset, as one bevel gear mounting arrangement is not restricted in the axial direction. 

\item Handle
\label{sec:org29d4d43}
A simple lego compatible handle is included to facilitate ergonomic device acutation. The handle incoporates a LEGO axle feature with an intermediate offset for easy assembly and dissassembly. The handle is fenestrated for asthetic and print time reduction.
\end{enumerate}

\subsubsection{Shell}
\label{sec:orgaff5752}
Shell components are designed for viewing accessibility, sufficient rigidity, and simple assembly/dissasembly. The shell assembly houses both the rotating split flap display assembly and drivetrain components. The shell is predominantly composed of two halves for printing simplicity and easy assembly.

\begin{enumerate}
\item Sides
\label{sec:org362a846}

The shell sides are of simple rectangular design, with mounting features for the central rotating split flap assembly, bevel gear and worm drive. Aside from these mounting features, shell side fenestration is maximized for viewing accessibility, without excessively sacrificing rigidty. Tight fit circular dog-bone cutouts are relieved from rectaungular extrusions on the top and bottom of each side of the assembly system, completed by dog bone connectors outlined below in section [ref dog bone connectors]. Shell sides are oriented with top and bottom protrusions extending away from the build plate, such that mechanism mounting features are all printed within the XY printing plane. The right shell half houses mating features for both the bevel gear and worm drive mounting components outlined in sections [ref bevel mount] and [ref worm drive mount]. These mating features are modelled with tight fits for both easy dissassembly and to ensure sufficiently consistent device actuation. Small rectagular tight fit cut-outs are provided symetrically in both shell sides for mounting the top and bottom stops.

Initial shell side designs did not exhibit satisfactory rigidity, allowing for excessive flex of the top flap stop, resulting in inconsistent flap acutation relative to rotational input. Fenestraitons were sucessively reduced until desired rigidity was acheived.  

\item Connectors
\label{sec:org84ac973}

The dog-bone connectors mirror the shell side dog-bone cut-outs. Overall connector length is 0.1mm less than the mating feature offsets in the assembled split flap display, effectively preloading the roatating splti flap assembly and creating the resistance required for smooth flap actuation. The connector profile is modelled using a tight fit offset with the mating shell side features, adding rigidity to the overall shell assembly, and holding connectors firmly in place without adhesive.

\item Stops
\label{sec:org4a3151c}
The top and bottom stops are an essential and deceptively complex component of the split flap display. Smoothness, consistency and the speed of flap actuation are all functions of the stop components. Different expereiments and adjustable stop assebmlies were employed to determine ideal stop position and thickness. [ref show early stop designs]. The final stop designs are sized using the experimental thickness and positions, which resulted in the best flap actuation.

\item Drivetrain Mounting
\label{sec:org0896266}
Drivetrain mounting components are printed seperately from the shell sides to ensure all axial rotational mounting features are printed wihtin the XY plane to ensure dimensional accuracy. Oblong hole deformation when printing YZ or XZ planes resulted in rough rotation and inconsistent flap actuation timing. All drivetrain components exhibit tight fit relationships with opposing shell side mounting features, achieving adequate rigidity, while maintaining the ability to dissassmble the drivetrain. List drivetrain components [ref exploded drawing].
\end{enumerate}

\subsection{Technical Drawings}
\label{sec:org2c1c70b}

\subsection{Exploded Diagrams}
\label{sec:orga913618}

\subsection{Photos}
\label{sec:orgcd1c263}

\subsection{Performance}
\label{sec:orgdae3635}
\end{document}