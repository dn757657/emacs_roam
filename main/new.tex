% Created 2022-07-20 Wed 09:22
% Intended LaTeX compiler: pdflatex
\documentclass[11pt]{article}
\usepackage[utf8]{inputenc}
\usepackage[T1]{fontenc}
\usepackage{graphicx}
\usepackage{longtable}
\usepackage{wrapfig}
\usepackage{rotating}
\usepackage[normalem]{ulem}
\usepackage{amsmath}
\usepackage{amssymb}
\usepackage{capt-of}
\usepackage{hyperref}
\date{\today}
\title{foam 3d printer}
\hypersetup{
 pdfauthor={},
 pdftitle={foam 3d printer},
 pdfkeywords={},
 pdfsubject={},
 pdfcreator={Emacs 28.0.91 (Org mode 9.5.2)}, 
 pdflang={English}}
\begin{document}

\maketitle
\tableofcontents


\section{Printers}
\label{sec:org9a8fd26}
\subsection{Prusa MK3s+ \url{https://www.prusa3d.com/product/original-prusa-i3-mk3s-kit-3/}}
\label{sec:org65525a6}

\subsubsection{Pros}
\label{sec:org4f74527}
\begin{itemize}
\item known for slicer quality
\item community
\item professional level tool
\item gold stndard
\item speed
\item enclosure available
\end{itemize}

\subsubsection{Cons}
\label{sec:orgeae14f9}
\begin{itemize}
\item long lead time 3-4 weeks
\end{itemize}

\subsection{Pulse 3D Printer  \url{https://www.matterhackers.com/store/l/pulse-3d-printer/sk/MH4C92XW?aff=7512}}
\label{sec:orgcd31e7f}
\subsection{Ender 3 S1 Pro \url{https://www.creality3dofficial.com/products/ender-3s1-pro-3d-printer}}
\label{sec:orgde1b792}

\section{Components}
\label{sec:orgfc7c4bd}
\begin{itemize}
\item \href{leveling_sensors.org}{leveling sensors}
\item \href{extruders.org}{extruders}
\end{itemize}

\section{Rating}
\label{sec:org491afe4}

\subsection{Hot Ends}
\label{sec:orgd4a3af9}
\begin{center}
\begin{tabular}{llllll}
Printer & Hot End & Max Temp. & Responsiveness/Material & Power & Options\\
\hline
Prusa & E3D V6 & 300C & Alu Copper & 30W & up to 500C w/upgrade\\
Ender & Sprite Pro & 300C & Alu & n/a & n/a\\
Pulse & E3D V6 & 300C & Alu Copper & 30W & up to 500C w/upgrade\\
 &  &  &  &  & \\
\end{tabular}
\end{center}

\begin{itemize}
\item heater materials are kind of ambiguous/not much information provided, can only assume that the copper alloy is going to be more responsive in terms of dissapating and transferring heat
\item Ender can be upgraded to E3D V6
\item just from breif googling E3D has much better documentation
\end{itemize}

\subsection{Extruder}
\label{sec:org9c09ebb}
\begin{center}
\begin{tabular}{lll}
Printer & Drive & Weight\\
Prusa & Dircet & \\
Ender & Direct & \\
Pulse & Bowden & \\
\end{tabular}
\end{center}

\begin{itemize}
\item weight can contribute to overshoot/ringing/ is generally bad
\item TPU and other flexible filaments can be printed using a bowden setup but not easily, dirct drive certainly much better
\end{itemize}

\subsection{Other}
\label{sec:org77d0710}
\begin{center}
\begin{tabular}{llllll}
Printer & Levelling Sensor & Documentation/Support & Community & Max Bed Temp & Printing Speed\\
\hline
Prusa & Inductive (pinda) & Good & Large & 120C & 80 mm/s\\
Ender & Hall Effect (CR Touch) & Not Great & Large & 110 C & 60 mm/s\\
Pulse & Hall Effect (BL Touch) & Exists & Exists & 115 C & 60 mm/s\\
 &  &  &  &  & \\
\end{tabular}
\end{center}

\begin{itemize}
\item hall effect levelling sensor may be a must with garolite printing bed and TPU adhesion, induction sensor still works but added bed thickness (above metal bottom) is limited
\end{itemize}

\subsection{Slicers}
\label{sec:org418e5ac}
\begin{itemize}
\item all slicers seem to be based on the same CUSA engine and are all open source
\item any slicer, since it is onyl producing G-CODE can be modified via code langauge of choice post process
\item Prusa slicer features preloaded presets for materials, etc, this is not seen as valuable in this instance as the project is using custom materials and possible slicer modificaitons
\end{itemize}

\subsection{Motion}
\label{sec:orgbfecc42}
\begin{itemize}
\item prusa is considered more accurate motion system, especially for small parts
\item prusa uses better bearings and not rubber wheels like ender
\end{itemize}

\subsection{Motherboard/Computing}
\label{sec:org67e9a46}
\begin{itemize}
\item both prusa and pulse are 8-bit, where ender is 32-bit
\end{itemize}

\section{Risks}
\label{sec:org756866d}

\subsection{Venting}
\label{sec:orgb1aadad}
\begin{itemize}
\item venting is required for most printing filaments
\item nothing crazy usually a fan out the window or similar
\item highly ventilated room is also acceptbable
\end{itemize}

\subsubsection{Materials}
\label{sec:org445264a}
\begin{itemize}
\item nylon (Caverna)
\item abs
\item TPU (varioshore)
\end{itemize}

\subsection{Hardware}
\label{sec:orgaf8880c}

\subsubsection{levelling}
\label{sec:orgb1fe14b}
\begin{itemize}
\item avoid inductive type leveling sensors as only works with metal bed materials, may not be suitable for filaments in mind \href{leveling_sensors.org}{leveling sensors}
\end{itemize}

\subsection{Misc}
\label{sec:org85cfdfa}
\begin{itemize}
\item delivery
\item ender hot end may not be responssive enough given aluminum only heating block
\item availability of correct bed surfaces for TPU adhesion, can also use glues on normal or glass beds
\item varioshore STARTS foaming at 230C, will 300C be enough to fully observe its properties?
\end{itemize}

\section{Important params}
\label{sec:org83be7a1}
\begin{itemize}
\item hot end control and response - titanium for ender both up to 300 degreesC
\item motion accuracy - prusa big advantage rubber vs solid and bedding
\item extruder both good but prusa not metla
\item prusa much higher motherbord - prusa may strruggle with tough g code
\item stepper drivers similar support sensorless homing only on prusa
\item prusa lightweight bed and auto level same bed on ender but manual levelling 110 deg ender vs 120 deg prusa
\item prusa best levelling sensor
\item ender has some dumb extras like lights
\item 

\item prua much more accurate
\item both can extrude the same filaments
\item prusa 80mm/min, 60mm/min for ender
\item 

\item prusa has much better software/firmware
\item prusa specific slicer
\item 

\item prusa wins big on documentation
\end{itemize}

\subsection{Extruders}
\label{sec:orge2efea3}

\subsection{Hot Ends}
\label{sec:org62928b7}

\subsubsection{Speed}
\label{sec:org13c13c9}

\subsection{Slicers}
\label{sec:orgf7fe5ce}

\subsection{Community/Support}
\label{sec:orgdf4dcc2}

\subsection{Leveling Sensors}
\label{sec:org161e788}

\subsection{Motion Accuracy}
\label{sec:orgccc8fb4}

\subsection{Motherboard/Computing}
\label{sec:orgac364bc}

\subsection{Bed}
\label{sec:org33223ec}

\subsection{Addons}
\label{sec:org153b1ad}
\end{document}